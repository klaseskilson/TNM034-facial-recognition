The aim of this project was to create a facial recognition system that would be able to recognize faces from a database of images. The images could have been modified and transformed in different ways to alter their appearances and compared with the database with successful results. Faces that didn’t exist in the database would also be tested against the system. If the system found a face, it would respond with the id of that person in the database. The system would also report if a face wasn’t recognized as being part of the database. 

Three image databases were given. One of the databases (DB1) was supposed to be used as the reference database – these were the images the system was supposed to be able to recognise. Another database contained images of persons not in the database (DB0). The third (DB2) contained images of persons in the reference database, but where certain parameters were changed; the light might be brighter or darker, the background could be cluttered, the person out of focus and the white balance could be different from the first database as well as the person’s facial expression. This database constitutes a great challenge, since these images are different from those in the first database. In order to allow these images to be recognised, the normalization process needs to be thorough.	