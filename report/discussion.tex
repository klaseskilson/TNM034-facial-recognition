The result obtained in our face recognition system with DB1 is dependent on the images being taken in an environment with no distracting background. With this problem in mind we were having problems getting any okay result at all on DB2, the database containing the challenging images of persons in the reference database. What happens is that the cluttered background contains skin-like parts that confuse the facial feature detection algorithm, and makes it think that there is skin where there is none. This leads to a situation where the skin detection, whose main purpose is to aid the eye and mouth detection, returns an unreasonably large mask. Within this large mask, there might be several eye and mouth candidates. An improvement of the implementation discussed in this paper could therefore be a better face detection method to easier mask out the areas with disturbance.

\subsection{Fisherfaces}
One method that could have been used instead of Eigenfaces to increase the success of our recognition is Fisherfaces. FIsherfaces rely on linear discriminant analysis (LDA), a method that succeed Fisher's linear discriminant method (hence the name Fisherfaces) and that recognises patterns in a set of images. Similar to PCA, this method focus a lot on dimensionality reduction. This method would replace the eigenfaces state, where the faces are recognised and not detected, of the system discussed here.

However, Fisherfaces is a method that greatly benefits from having several photos of the same class, in our case image of a face. Here, Fisherfaces focuses on maximizing the ratio of the between-class differences and the within-class differences (Eigenfaces vs. Fisherfaces: Recognition using class specific linear projection). Since we do know which class, again; face, each image belong to, we can use images from DB2 to create a better performing Fisherface database. This method can be considered more complex than eigenfaces and relies on a solid and reliable facial detection system.

According to Eigenfaces vs. Fisherfaces: Recognition using class specific linear projection, Eigenfaces has a higher error-rate when parameters such as lightning and expression vary. Therefore, the system is likely to benefit from an implementation of Fisherfaces. Other than the increased complexity, there isn’t really anything negative with Fisherfaces compared to Eigenfaces.

\subsection{LogAbout}
LogAbout was implemented and is available as an extra normalization feature in our program. However it was hard to find any values that worked out for our dataset. The problem we found was that it was hard to determine what values the constants a, b, and c should have in the equation. In the researched papers for the implementation of the function there were no given constants since every dataset has its own values that fit for it.  Therefore attempts were made to determine the constants by trial and error. There were no good results that came out of this and it was settled that the LogAbout method were not to be used in our implementation since it only worsened the final result.
