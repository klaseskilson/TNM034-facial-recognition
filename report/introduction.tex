The art of detecting faces and recognizing them is a part of our everyday life. Facial recognition is used as an authentication system for our phones and laptops. With the increasing personal information stored on these devices, the need for the security and reliance of the recognition has to be increased as well. We also expect the recognition to work with any conditions in regard of lighting conditions and orientation. Within the research area of image processing, these implication of these obstacles can be eliminated or their effect decreased.

This project applied some of these techniques in the implementation of a facial recognition system with mixed result in regards of false rejection rate (FRR) and false acceptance rate (FAR). FRR measures the rate of which a face in the database is not recognized while FAR measures the incorrect of the wrong person in the database or a person not in the database at all.

